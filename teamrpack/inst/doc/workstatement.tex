\documentclass[12pt,letterpaper]{article}

\usepackage{amsmath, amsthm, amssymb, amsfonts}
\usepackage{graphicx}
\usepackage{bm}
\usepackage{natbib}

\theoremstyle{definition}
\newtheorem{dfn}{Definition}

\begin{document}

% The numbers below controls the amount of space between the following sections
\def\shiftdowna{0.32in}  % Adjust for balance
\def\shiftdownb{0.22in}  % Adjust for balance

% Set up the boiler plate at the top of the page

\begin{center}
\textbf{{\large Project Work Statement}}\\


% SPONSOR
\vspace \shiftdowna
\underline {Sponsor}\\ 
\vspace{5pt}
\textsc{{\large Renaissance Technologies LLC}}\\


% TITLE
\vspace \shiftdowna
\textbf{{\large A Study on Statistical Arbitrage}}


% STUDENTS
\vspace{0.35in}
\vspace \shiftdownb
\underline {Participants} \\
\vspace{5pt}
\text{Huinan Zhang}, \texttt{zhang.huinan@gmail.com}\\
\text{Jing Huang}, \texttt{hjing.zju@gmail.com}\\
\text{Rong Fan}, \texttt{fantasy0409@gmail.com}\\
\vspace{3pt}




% SPONSORS
\vspace \shiftdownb
\underline {Project Advisor}\\
\vspace{5pt}
\text{Nam Lee}, \texttt{nhlee@jhu.edu}

% DATE
\vspace \shiftdowna
Date: \today

\end{center}

\vfill  
%Fill page to force following note to bottom
\footnoterule
\noindent \small{Any apparent association of this work to Renaissance
  Technologies is fictional, and the sole purpose of this work is a
  class exercise.}

\newpage

\section{Background} 

Renaissance Technologies is one of the leading, though not highly
publicized, hedge fund. It uses sophisticated mathematical models and
highly-optimized computer programs to carry out trading
activities. According to a report on Wall Street Journal in 2005, the
company has achieved a cumulative return of more than 2000\% of the
the 11 years since its founding till 1999.  Renaissance Technologies
currently have more than \$23 billion in assets under management. The
company carries out its trading activities globally across a large
variety of financial products. In terms of return performance,
Renaissance Technologies is among the best hedge funds in the
world. It maintains its advantage by actively conducting research on
the market and trading strategies.


\section{Problem Statement}

Among the numerous trading transactions aimed at maximizing the return
on investment, the most favorable ones are in the form of
arbitrage. Arbitrage is the practice of making risk-less profit
without cost. It takes advantage of the imperfections in the financial
market as reflected by inadequate pricing of financial products.


The difficulty of exercising arbitrage often lies in finding such
opportunities and executing the transactions quickly before such
opportunities disappear. It is our goal to explore the mechanisms of
statistical arbitrage in order to help us find arbitrage
opportunities. However, we will not be able to execute the
trade. Instead, we estimate the time windows within which arbitrage
opportunities exist. These time windows will serve as constraints for
real order execution.






\section{Approach}

Our study begins with a literature review of existing arbitrage
techniques. Our major source of information on the mechanism is
academic papers. Based on existing papers, we will create a
mathematical model for us to identify arbitrage opportunities. We will
devise a trading strategy using the mathematical model to carry out
trading.


Our arbitrage trading strategy will be programmed in code, and
back-tested using historical financial data. We will divide our
testing data into two part: the training set and the test set. We plan
to use around 12 months' data as training set and the 3 months' data
immediately following them as test test. The performance of our
program will be measured by the number of arbitrage opportunities
found, the feasibility of the arbitrage measured as the length of the
execution time window, and ultimately the potential return on
investment if our trading program were used in practice.




\section{Milestones}
We have the following major deadlines:
\begin{itemize}
    \item Work Statement due date, Sep 28, 2012,
    \item Midterm Presentation due date, Oct 12, 2012,
    \item Progress Report due date, Oct 26, 2012,
    \item Final Presentation due date, Nov 6, 2012,
    \item Final Report due date, Nov 30, 2012.
\end{itemize}

\section{Deliverable}
\subsection{From Team to Sponsor} % (fold)
The following outputs are expected from this project:
\begin{itemize}
\item A trading program that implements statistical arbitrage
\item The data for testing of the program
\item A technical report providing the details of the trading program
  and the result of the testing
\end{itemize}

\subsection{From Sponsor to Team} % (fold)

The corporate regulations on information disclosure forbids the
dissemination of any source code or data to a party that is not
formally affiliated with Renaissance Technologies. We will only use
information sources that is available publicly and the ones that are
available to Johns Hopkins University students.


%\newpage
%\bibliographystyle{plain}
%%\renewcommand\bibname{Selected Bibliography Including Cited Works}
%\nocite{*}
%\bibliography{biblio}

\end{document}
